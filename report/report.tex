\documentclass[12pt]{article}
\usepackage{makeidx}
\usepackage[margin=1in]{geometry}  % set the margins to 1in on all sides
\usepackage{graphicx}              % to include figures
\usepackage{amsmath}               % great math stuff
\usepackage{amsfonts}              % for blackboard bold, etc
\usepackage{amsthm}                % better theorem environments
\usepackage{makeidx}               % index
\usepackage[utf8]{inputenc}        % now we have tildes!
\usepackage{graphicx}
\usepackage{wrapfig}
\usepackage{listings}
% \usepackage[spanish]{babel}

% various theorems, numbered by section

\graphicspath{{img/}}

\begin{document}

\begin{titlepage}

\newcommand{\HRule}{\rule{\linewidth}{0.5mm}} % Defines a new command for the horizontal lines, change thickness here

\center % Center everything on the page

%----------------------------------------------------------------------------------------
%	LOGO SECTION
%----------------------------------------------------------------------------------------
% \noindent \includegraphics[width=\textwidth]{class_or_no}
\includegraphics[width=0.5\textwidth]{media/uwlogo}\\[1cm] % Include a department/university logo - this will require the graphicx package

%----------------------------------------------------------------------------------------

%----------------------------------------------------------------------------------------
%	HEADING SECTIONS
%----------------------------------------------------------------------------------------

% \textsc{\LARGE University of Waterloo}\\[3.5cm] % Name of your university/college
\textsc{\Large Computer Science}\\[0.5cm] % Major heading such as course name
\textsc{\large Concurrent and Parallel Programming}\\[2.5cm] % Minor heading such as course title

%----------------------------------------------------------------------------------------
%	TITLE SECTION
%----------------------------------------------------------------------------------------

\HRule \\[0.4cm]
{ \huge \bfseries Assignment 1: Exceptions and Semi-Coroutines}\\[0.4cm] % Title of your document
\HRule \\[4.5cm]

%----------------------------------------------------------------------------------------
%	AUTHOR SECTION
%----------------------------------------------------------------------------------------


% If you don't want a supervisor, uncomment the two lines below and remove the section above
\emph{Author:}\\[0.7cm]

\begin{tabular}{rl}
    Daniel \textsc{Medina García}: &\texttt{dmedinag@uwaterloo.ca}\\
\end{tabular}\\[2cm]


%----------------------------------------------------------------------------------------
%	DATE SECTION
%----------------------------------------------------------------------------------------

{\large \today}\\[3cm] % Date, change the \today to a set date if you want to be precise



\vfill % Fill the rest of the page with whitespace

\end{titlepage}

\tableofcontents % Table of contents page

\newpage

\section{Code description} % Content starts here

The provided source code is divided in three files, one per exercise.

\subsection{Source code}

\newpage

\subsection{Descripción del Código de \texttt{RRFI.c}}

\newpage

\section{Test and analysis of results}

\newpage

\section{Conclusions and Problems Faced}

\end{document}
